\documentclass{beamer}


\setbeamertemplate{background canvas}[vertical shading][bottom=red!10,top=blue!10]

\usetheme{Warsaw}
%\usecolortheme{crane}
\usefonttheme[onlysmall]{structurebold}
\setbeamerfont{title}{shape=\itshape,family=\rmfamily}
%\setbeamercolor{title}{fg=red!80!black}
\usepackage{tikz}
\usepackage{pgf}
\usepackage{amsmath,amssymb}
\usepackage{tabularx}
\usepackage{supertabular}
\usepackage[latin1]{inputenc}
\usepackage{gensymb}

\usepackage{colortbl}
\usepackage{listings}
\setbeamercovered{dynamic}
%\usepackage[spanish]{babel}
\usepackage[spanish,english]{babel}
\selectlanguage{english}

\title[NUDFT]{Non-uniform Discrete Fourier Transform}
\author[R. Pezoa]{%
  Raquel~Pezoa}
\institute[Universidad T\'ecnica Federico Santa Mar\'ia]{
  %
  Departamento de Inform\'atica\\
  Universidad T\'ecnica Federico Santa Mar\'ia
}
\date[Enero 2012]{Enero 2012}
\subject{Seminario}



%\pgfdeclareimage[opciones]{nombre de imagen}{nombre del archivo, sin extension}
\pgfdeclareimage[width=2cm]{pierreBezier}{img/pierreBezier}
%\newtheorem{theorem2}{Teorema}
%\newtheorem{example2}{Ejemplo}
%\newtheorem{definition2}{Definición} 


\begin{document}
  
  \frame{\titlepage}
  
% \section*{Outline} 
% \begin{frame}
%   \frametitle{Outline}
%   \tableofcontents
% \end{frame}

\begin{frame}
\begin{center}
These slides are based on the course:


\end{center}
\end{frame}
  

\begin{frame}
\frametitle{Fourier}
A simple reminder:
\begin{itemize}
\item Fourier Series:
$$f(t)=\sum_{k=0}^{\infty}(A_{k}\cos{\omega_{k}t} + B_{k}\sin\omega_{k}t)$$

\item Fourier Series in Complex Notation:
$$f(t)=\sum_{k=-\infty}^{k=+\infty}C_{k}e^{i\omega_{k}t}$$


\item Continuous Fourier Transform:
$$ F(\omega) = \frac{1}{2\pi} \int_{-\infty}^{+\infty} f(t)e^{-i \omega t}dt$$ 


\end{itemize}
\end{frame}

\begin{frame}
\frametitle{Discrete Fourier Transform (DFT)}
\begin{itemize}

\item Often, we do not know a function's continous
  \emph{``behaviour''} over time, but only what happens at $N$
  discrete times:
$$t_{k}= k\Delta t, \ \ k=0,1,2,\ldots , N-1 \ (regular sampling!)$$ 

$$e^{i\omega t} \rightarrow e^{i \frac{2\pi t_{k}}{T}}  \rightarrow  e^{\frac{2\pi i k \Delta t}{N \Delta t}} = e^{\frac{2 \pi i k }{N}}$$

$$F_{j} = \frac{1}{N} \sum_{k=0}^{N-1}f_{k}e^{\frac{-j 2 \pi k n}{N}}  $$

\end{itemize}
\end{frame}


\begin{frame}
\frametitle{Non-uniform Discrete Fourier Transform (NDFT)}
\begin{itemize}
\item When the samples are irregularly taken in the time
domain $t$ but regularly taken in the frequency domain.

\item That is to say that the samples $F_{j}$ of the irregular Fourier
  transform are taken at multiples of a quantity $\Delta k$, which is
  a fixed quantity in the Fourier domain ($\frac{k=2 \pi}{T}$, for the
  regular case).

\item NUDFT:
$$F_{j}=\sum_{n=0}^{N-1}f_{n}e^{-i\frac{2 \pi}{T}m t_{n}}$$

\end{itemize}

\end{frame}

\end{document}


